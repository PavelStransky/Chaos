\documentclass[a4paper,11pt,twoside]{article}
\usepackage[utf8]{inputenc}	%% Text coding
\usepackage[czech]{babel}
\usepackage{epsfig,subfigure}
\usepackage{amsfonts,amsmath,amssymb}
\usepackage{graphicx}
\usepackage[unicode]{hyperref}
\usepackage{indentfirst}
\usepackage{fancyhdr}
\usepackage{xifthen}
\usepackage{amsthm,thmtools}
\usepackage{bold-extra}

\hypersetup{
	pdftitle={Classical and Quantum Chaos},
	pdfauthor={Pavel Stránský},
	pdffitwindow=true,
	colorlinks=true,
	urlcolor=cyan,            %barva textu pri tisku
	linkcolor=red,
	citecolor=green,
	filecolor=magenta
}

% Velikost stránky
\addtolength{\topmargin}{-1.5cm} %\addtolength{\textheight}{-10cm}
\addtolength{\textwidth}{4cm} \addtolength{\textheight}{4cm} % Šířka a výška textu
\addtolength{\voffset}{-0.5cm} % Horní okraj
\addtolength{\hoffset}{-2cm}
\setlength{\headheight}{15pt}

\pagestyle{fancy}

\DeclareMathOperator{\e}{e}

\def\vector#1{\boldsymbol{#1}}								% Vector
\renewcommand{\d}{\mathrm{d}}
\newcommand{\derivative}[3][]{\ifthenelse{\isempty{#1}}	    % Normal derivative
	{\frac{\d{#2}}{\d{#3}}}
	{\frac{\d^{#1}{#2}}{\d{#3}^{#1}}}
}
\def\makematrix#1{\begin{pmatrix}#1\end{pmatrix}}       % Matrix

\def\code#1{\textnormal{\texttt{#1}}}
\def\file#1{\textnormal{\textbf{\texttt{#1}}}}

\newtheoremstyle{spaced}
{5pt}{5pt}{\itshape}{}{\bfseries}{:}{.5em}{}

\newtheoremstyle{red}
{5pt}{5pt}{\itshape\color{red}}{}{\bfseries\color{red}}{:}{.5em}{}

\newtheoremstyle{blue}
{5pt}{5pt}{\itshape\color{blue}}{}{\bfseries\color{blue}}{:}{.5em}{}

\begin{document}
\theoremstyle{spaced}
\newtheorem{example}{Příklad}[section]

\theoremstyle{red}
\newtheorem{task}{Úkol}[section]

\theoremstyle{blue}
\newtheorem{solution}{Řešení}[section]

\title{Lecture Notes on Quantum Chaos}
\date{\today}
\author{Pavel Stránský}

\maketitle

\section{Literature}
\begin{itemize}
    \item \cite{Gut90}~Martin C. Gutzwiller, {\it Chaos in Classical and Quantum Mechanics} (Springer, 1990)
        \begin{itemize}
            \item Classical monography about chaos in physics.
            \item Profound physical and mathematical discussion.
            \item Sometimes nonstandard notation.
        \end{itemize}

    \item \cite{Haa10}~Fritz Haake, {\it Quantum Signatures of Chaos} (Springer, 2010)
        \begin{itemize}
            \item Up-to-date topics, including chaotic dissipative systems and supersymmetric approaches.
        \end{itemize}

    \item \cite{Boh89}~Oriol Bohigas, {\it Random Matrix Theories and Chaotic Dynamics}, Les Houches LII, ed. M.-J. Gianonni, A. Voros, J. Zinn-Justin, 1989
        \begin{itemize}
            \item Brief and conscise notes on the basics of quantum chaos.
        \end{itemize}

    \item \cite{Meh04}~Madan L. Mehta, {\it Random Matrices} (Elsevier 2004).
        \begin{itemize}
            \item Everything you always wanted to know about random matrices (and quantum chaos is from a big part about random matrices).
            \item \dots~and probably even didn't want to know.
            \item If you love formulae, you'll be happy happy reading this book.
        \end{itemize}

    \end{itemize}

\section{A way to quantum chaos}
    \subsection{Quantum mechanics is linear}
        Classical chaos is tightly connected with the nonlinearity of classical equations of motion.
        Quantum mechanics is linear, quantum evolution unitary$\longrightarrow$no sensitive dependence on ``initial conditions''.


        

\begin{thebibliography}{99}
    \bibitem{Gut90} Martin C. Gutzwiller, {\it Chaos in Classical and Quantum Mechanics} (Springer, 1990).
    \bibitem{Haa10} Fritz Haake, {\it Quantum Signatures of Chaos} (Springer, 2010).
    \bibitem{Boh89} Oriol Bohigas, {\it Random Matrix Theories and Chaotic Dynamics}, Les Houches LII, ed. M.-J. Gianonni, A. Voros, J. Zinn-Justin, 1989.
    \bibitem{Meh04} Madan L. Mehta, {\it Random Matrices} (Elsevier 2004).

    \bibitem{Rei04} Linda E. Reichl, {\it The Transition to Chaos: Conservative Classical Systems and Quantum Manifestations} (Springer, 2004).
    \bibitem{Sto99} Hans-Jürgen Stöckmann, {\it Quantum Chaos: An Introduction} (Cambridge University Press, 1999).
    \bibitem{Ozo88} Alfredo M. Ozorio de Almeida, {\it Hamiltonian Systems: Chaos and Quantization} (Cambridge University Press, 1988).
\end{thebibliography}

\end{document}
