\documentclass[a4paper,11pt,twoside]{article}
\usepackage[utf8]{inputenc}	%% Text coding
\usepackage[czech]{babel}
\usepackage{epsfig,subfigure}
\usepackage{amsfonts,amsmath,amssymb}
\usepackage{graphicx}
\usepackage[unicode]{hyperref}
\usepackage{indentfirst}
\usepackage{fancyhdr}
\usepackage{xifthen}
\usepackage{amsthm,thmtools}
\usepackage{bold-extra}

\hypersetup{
	pdftitle={Classical and Quantum Chaos},
	pdfauthor={Pavel Stránský},
	pdffitwindow=true,
	colorlinks=true,
	urlcolor=cyan,            %barva textu pri tisku
	linkcolor=red,
	citecolor=green,
	filecolor=magenta
}

% Velikost stránky
\addtolength{\topmargin}{-1.5cm} %\addtolength{\textheight}{-10cm}
\addtolength{\textwidth}{4cm} \addtolength{\textheight}{4cm} % Šířka a výška textu
\addtolength{\voffset}{-0.5cm} % Horní okraj
\addtolength{\hoffset}{-2cm}
\setlength{\headheight}{15pt}

\pagestyle{fancy}
% Definice
\DeclareMathOperator{\e}{e}
\DeclareMathOperator{\tg}{tg}
\DeclareMathOperator{\cotg}{cotg}
\DeclareMathOperator{\arccotg}{arccotg}
\DeclareMathOperator{\sign}{sign}
\DeclareMathOperator{\arccosh}{arccosh}
\DeclareMathOperator{\arcsinh}{arcsinh}
\DeclareMathOperator{\divergence}{div}
\DeclareMathOperator{\gradient}{grad}
\DeclareMathOperator{\trace}{Tr}
\DeclareMathOperator{\real}{Re}
\DeclareMathOperator{\imaginary}{Im}

\renewcommand{\d}{\mathrm{d}}

\def\O#1{\mathcal{O}\left({#1}\right)}

\def\ket#1{\left|{#1}\right\rangle}
\def\bra#1{\left\langle{#1}\right|}
\def\mean#1{\left\langle{#1}\right\rangle}
\def\braket#1#2{\left\langle{#1}\middle|{#2}\right\rangle}
\def\matrixelement#1#2#3{\left\langle{#1}\middle|{#2}\middle|{#3}\right\rangle}
\def\ketbra#1#2{\left|{#1}\middle\rangle\middle\langle{#2}\right|}
\def\projector#1{\left|{#1}\middle\rangle\middle\langle{#1}\right|}

\def\clebsch#1#2#3#4#5#6{\mathcal{C}^{#5\,#6}_{#1\,#2\:#3\,#4}}
\def\threej#1#2#3#4#5#6{\begin{pmatrix}#1&#2&#3\\#4&#5&#6\end{pmatrix}}

\def\commutator#1#2{\left[{#1},{#2}\right]}
\def\associator#1#2#3{\left[{#1},{#2},{#3}\right]}

\def\abs#1{\left|{#1}\right|}
\def\abss#1{\left|{#1}\right|^{2}}							% Square of the absolute value
\def\intinf{\int_{-\infty}^{\infty}}						% Infinite integral

\def\minus#1{\left(-1\right)^{#1}}
%\def\ui#1{(#1)}
\def\ti#1{\mathrm{#1}}										% Text index

\def\error#1{{\color{red}{\bf{#1}}}}
\def\trick#1{{\color{blue}#1}}

\def\hilbert#1{\mathcal{#1}}								% Hilbert space
\def\group#1{\mathrm{#1}}									% Group
\def\algebra#1{\mathrm{#1}}

\def\vector#1{\boldsymbol{#1}}								% Vector
\def\matrix#1{\mathsf{#1}}										% Matrix
\def\axis#1{\mathrm{#1}}

\def\2F1#1#2#3#4{\,{}_{2}F_{1}\!\left(#1,#2,#3;#4\right)}
\def\1F1#1#2#3{\,{}_{1}F_{1}\!\left(#1,#2;#3\right)}

\def\operator#1{\mathsf{\hat{#1}}}
\def\vectoroperator#1{\boldsymbol{\mathsf{\hat{#1}}}}
\def\tensoroperator#1#2{\hat{\mathbb{#1}}^{(#2)}}					% tensor operator
\def\tensoroperatorcomponent#1#2#3{\hat{\mathsf{#1}}^{(#2)}_{#3}}	% tensor operator - component
\def\reducedmatrixelement#1#2#3{\left(#1\middle\lVert#2\middle\rVert#3\right)}	    % Reduced matrix element

\def\propagator{G(\vx_{\rf},t_{\rf};\vx_{\ri},t_{\ri})}

\newcommand{\partialderivative}[3][]{\ifthenelse{\isempty{#1}}	% Partial derivative
	{\frac{\partial{#2}}{\partial{#3}}}
	{\frac{\partial^{#1}{#2}}{\partial{#3}^{#1}}}
}

\newcommand{\derivative}[3][]{\ifthenelse{\isempty{#1}}	% Normal derivative
	{\frac{\d{#2}}{\d{#3}}}
	{\frac{\d^{#1}{#2}}{\d{#3}^{#1}}}
}

\def\conjugate#1{{#1}^{\dagger}}
\def\transpose#1{{#1}^{\intercal}}

\def\makematrix#1{\begin{pmatrix}#1\end{pmatrix}}       % Matrix
\def\Vdots{\vphantom{0}\smash[t]{\vdots}}

\def\equationcomment#1{\begin{vmatrix}#1\end{vmatrix}}  % Comment in equation (e.g. substitution in integral)

\def\important#1{\boxed{#1}}

\def\MeV{\mathrm{MeV}}
\def\im{\mathrm{i}}
\def\const{\mathrm{const}}



\begin{document}
\title{Lecture Notes on Quantum Chaos}
\date{\today}
\author{Pavel Stránský}

\maketitle

\section{Literature}
\begin{itemize}
    \item \cite{Gut90}~Martin C. Gutzwiller, {\it Chaos in Classical and Quantum Mechanics} (Springer, 1990)
        \begin{itemize}
            \item Classical monography about chaos in physics.
            \item Profound physical and mathematical discussion.
            \item Sometimes nonstandard notation.
        \end{itemize}

    \item \cite{Haa10}~Fritz Haake, {\it Quantum Signatures of Chaos} (Springer, 2010)
        \begin{itemize}
            \item Up-to-date topics, including chaotic dissipative systems and supersymmetric approaches.
        \end{itemize}

    \item \cite{Boh89}~Oriol Bohigas, {\it Random Matrix Theories and Chaotic Dynamics}, Les Houches LII, ed. M.-J. Gianonni, A. Voros, J. Zinn-Justin, 1989
        \begin{itemize}
            \item Brief and conscise notes on the basics of quantum chaos.
        \end{itemize}

    \item \cite{Meh04}~Madan L. Mehta, {\it Random Matrices} (Elsevier 2004).
        \begin{itemize}
            \item Everything you always wanted to know about random matrices (and quantum chaos is from a big part about random matrices).
            \item \dots~and probably even didn't want to know.
            \item If you love formulae, you'll be happy happy reading this book.
        \end{itemize}

    \end{itemize}

\section{A way to quantum chaos}
    \subsection{Quantum mechanics is linear}
        Classical chaos is tightly connected with the nonlinearity of classical equations of motion.
        Quantum mechanics is linear, quantum evolution unitary$\longrightarrow$no sensitive dependence on ``initial conditions''.

\begin{thebibliography}{99}
    \bibitem{Gut90} Martin C. Gutzwiller, {\it Chaos in Classical and Quantum Mechanics} (Springer, 1990).
    \bibitem{Haa10} Fritz Haake, {\it Quantum Signatures of Chaos} (Springer, 2010).
    \bibitem{Boh89} Oriol Bohigas, {\it Random Matrix Theories and Chaotic Dynamics}, Les Houches LII, ed. M.-J. Gianonni, A. Voros, J. Zinn-Justin, 1989.
    \bibitem{Meh04} Madan L. Mehta, {\it Random Matrices} (Elsevier 2004).

    \bibitem{Rei04} Linda E. Reichl, {\it The Transition to Chaos: Conservative Classical Systems and Quantum Manifestations} (Springer, 2004).
    \bibitem{Sto99} Hans-Jürgen Stöckmann, {\it Quantum Chaos: An Introduction} (Cambridge University Press, 1999).
    \bibitem{Ozo88} Alfredo M. Ozorio de Almeida, {\it Hamiltonian Systems: Chaos and Quantization} (Cambridge University Press, 1988).
\end{thebibliography}

\end{document}
